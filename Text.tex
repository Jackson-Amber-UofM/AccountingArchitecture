\documentclass[12pt]{article}
\usepackage[left=1in, right=1in, top=1in, bottom=1in]{geometry}
\usepackage{color}
\usepackage{setspace}
\usepackage{graphicx}
\tolerance=1
\emergencystretch=\maxdimen
\hyphenpenalty=10000
\hbadness=10000
\frenchspacing{}
\clubpenalty 10000
\widowpenalty 10000
\newcommand{\AuthorsNote}[1]{\marginpar{\color{red}\tiny{}\raggedright{}\singlespace{}#1}}
\newcommand{\Citation}[0]{{\color{red}(citation)}}
\newcommand{\Abstract}[0]{\centerline{\textbf{\MakeUppercase{Abstract}}}\medskip{}}
\newcommand{\Section}[2]{\vspace{.2in}\centerline{\normalsize{}\textbf{#1\quad{}\MakeUppercase{#2}}}\nopagebreak{}\smallskip{}\indent{}}
\newcommand{\SubSection}[1]{\vspace{.15in}{\centering{}\normalsize{}\textbf{#1}}\\*\indent{}}
\newcommand{\SubSubSection}[1]{{\centering{}\normalsize{}\textbf{\emph{#1}}}\\*\indent{}}
\newcommand{\InsertGraphic}[1]{\vspace{.1in}\noindent{}\centerline{\MakeUppercase{---\quad{}Insert #1 here\quad{}---}}\vspace{.1in}}
\newcommand{\Reference}[1]{\parindent=0pt\hangindent=3em\hangafter=1#1\vspace{.15in}}
\begin{document}

\doublespace{}
\thispagestyle{empty}
{\centerline{\large{}\textbf{Accounting Architecture: The New Face of AIS}}}\vspace{.1in}
\Abstract{}
Unlike Financial Accounting, Managerial Accounting, Tax Accounting and Auditing, Accounting Information Systems as an academic discipline is not currently focused on training students to enter a discrete professional field. This lack of focus is not demand-driven as an increasingly dire need exists for IT auditors, as well as systems developers with accounting knowledge. One hurdle preventing expansion of AIS curriculum to prepare accounting students to serve as IT auditors and IT architects is the need for a unifying framework that circumscribes relevant professional competencies. This article introduces such a framework and proposes a corresponding academic discipline that combines accounting and auditing with systems design and control: Accounting Architecture.\\[.2in]
\textbf{JEL Codes:} M4, L8\\[.1in]
\textbf{Keywords:} Accounting Information Systems, Information Life Cycle, Technology, Control, Accounting Education\\[.2in]
\newpage{}
\setcounter{page}{1}

\Section{I.}{Introduction}
In this paper, we introduce a new framework for accounting education that focuses on information creation and maintenance, data analysis and IT infrastructure that we call Accounting Architecture (AA). This new framework responds to the call by employers for accountants to increase their understanding of current technologies and data analytics (Deloitte 2013; AACSB 2014; PWC 2015). By combining knowledge across the disciplines of accounting and information systems, information technology and computer science, the AA framework provides a unifying view of what constitutes accounting information and explains how financial, tax and management accounting, audit and internal controls, business intelligence, programming and IT all fit into one system. Because this framework incorporates these principles in the context of one another, it can help prepare accountants to work as systems designers, data analysts and IT auditors. Accountants with IT and data analytics expertise occupy a unique position to guide corporate strategy by creating a link between business activities and the IT functions that support those activities.

The goal of this framework is not to train accountants to replace IT specialists, but rather to provide accountants with sufficient understanding of information and technology to be able to actively participate in IT and IT audit roles. A recent article explained that IT innovation has increased the need for CIOs and their teams to work with business counterparts.\footnote{https://enterprisersproject.com/article/2015/5/cios-should-make-sure-their-teams-are-strong-their-networks} Accountants, as custodians of business information and reporting, are their business counterparts and must increase their ability to communicate with and assist IT teams. A former chairman of the New York State Society of Certified Public Accountant's Technology Assurance Committee observed that the current status quo was to hire information systems students to perform even the most menial IT audit and data analysis tasks because accountants lacked relevant knowledge. He went on to add that if we continued to refuse to learn their skills, IT specialists would learn ours and replace us. However, accountants are well positioned to overcome this risk. As a member of PWC internal audit stated, ``We prefer to train accountants to do IT tasks, than to train IT specialists to understand double-entry accounting.'' The Accounting Architecture framework proposed in this paper can instill in accountants the necessary abilities to meet the current demands placed on the profession.

The rate of technological innovation has exploded in recent years, and with this increase has come unprecedented corporate investment in IT infrastructure. One of the motivations for this new level of IT adoption is the recent manifestation of vast amounts of structured and unstructured data and the understanding that, if properly analyzed and understood, this data would provide decision makers with insights into new strategies.

As a result, data analytics has become a top strategic priority.\footnote{http://blogs.wsj.com/cio/2014/02/06/cios-rank-analytics-as-top-strategic-priority/} Firms collect new data each day, and the process of data creation has accelerated to the point that the world creates 2.5 quintillion bytes of data daily.\footnote{http://www-01.ibm.com/software/data/bigdata/what-is-big-data.html} More than 90 percent of all the data in the world is less than five years old (SINTEF 2013). The existence of this vast pool of structured and unstructured data (i.e., Big Data) has encouraged firms to invest in data analysis tools to learn about customer behavior, economic trends or any other insight the data can provide to create a competitive edge.

New storage, analysis and cloud computing tools have facilitated the creation, collection and use of Big Data. However, the emergence of new technologies requires new expertise. As a result, companies increasingly demand individuals with expertise in Linux,\footnote{http://readwrite.com/2014/08/20/linux-jobs-demand-certification} CloudStack, NoSQL and Hadoop.\footnote{http://www.opensourceforu.com/2014/05/foss-skills-will-get-hired-year/} For example, in its recent white paper, PWC has demanded specific skill sets of future accountants (2015). Not surprisingly, some of these skills overlap with those demanded by companies at large, specifically NoSQL and Hadoop, but PWC also lists SQL, Python, R and Tableau as important skills. A solid understanding of these technologies would enable an accountant to collect, transform, analyze and visualize data.\footnote{It is likely that the decision to list R and not SAS or Python and not PERL is not an attempt by PWC to elevate one technology over another. Instead, these should represent example technologies for specific skill sets.}

Currently, accounting education focuses almost exclusively on accounting regulation and compliance. Although this focus is beneficial because of the need to have a broad understanding of accounting rules, it fails to instill in students an understanding of the information system as a whole. Even those courses (i.e., management accounting and accounting information systems) that teach topics other than accounting rules and regulations emphasize business cycles, segregation of duties and cost formulas at the expense of providing perspective into the process by which data becomes information.

One prominent exception is the frequency of Excel and Access training. Although the majority of technologies listed by PWC are not office software, PWC also lists these two as ``legacy technologies.'' The mention of Excel and Access highlights two important points. First, the pursuit of new expertise should not immediately supplant expertise in existing software. Excel remains a widely used business tool, and any loss of proficiency with it robs the accounting profession of some ability to master the information system. Second, different firms have different systems and different needs. Hadoop is a great tool for organizing data, but not all companies have petabytes of unstructured data. Some need only a small relational database or a few Excel spreadsheets or QuickBooks. The popularity of new, enterprise-grade data analysis tools should not negate the importance of less complex software and storage solutions.

Because employers have begun to demand new skills of accountants, the academy should reevaluate the current model for accounting information systems and the encompassing curriculum to determine what revisions are necessary to train accountants with these requisite skills. The need for data analysts should represent only a starting point for identifying curriculum improvements. Many companies have expressed the desire for data analysts, but it is important not to allow this urgency for data analysis to cause the academy or the profession to miss the mark. One of the problems with the earlier wide adoption of ERP systems was that companies felt an urgency to have an ERP system. Without an understanding of information needs, benchmarks were unavailable to measure whether the ERP system would fill those needs. This same fever now exists for Big Data, but in order for any addition to the information system to be useful, the participants must understand the information system as a whole.

This paper proceeds as follows. In the next section, we contrast the existing AIS model and its implementation in current curricula with the new Accounting Architecture framework. In section three, we decompose the new model and explain its individual components. In section four, we suggest curriculum modifications. In section five, we conclude.

\Section{II.}{Model Comparison}
Figure 1 displays a generalized AIS model prevalent in current AIS textbooks. This model includes three processes---data collection, data processing and information reporting---and one object---data storage. Data collection is the receipt of external sources of data. Data processing updates data storage with newly collected data and uses data in storage to calculate statistics. Information reporting presents statistics in a format that is accessible to decision makers. Data storage represents accounting journals and ledgers.

\InsertGraphic{Figure 1}

The steps in this model differ from those in the information life cycle (i.e., the process by which data becomes usable information) as understood by information scientists.\footnote{Some information scientists refer to the information life cycle by its pre-digital name: records life cycle or records management life cycle.} Because of the complexities of managing digital materials in their many forms, information scientists perceive a more extensive life cycle comprising: needs assessment, data acquisition, classification, conversion, data entry, storage, organization, indexing, rights management, refreshing, interpretation, disposition, searching, data analytics and reporting (Dederer and Dmytrenko 2015).

Current curricula use the existing AIS model to teach the process by which business cycles generate general ledger information. These cycles, the revenue cycle, expenditure cycle and conversion cycle, are the primary focus of AIS textbooks. A primary reason for teaching cycles is that wide-spread ERP adoption necessitated uniform business processes across firms (Scheer and Habermann 2000; Jacobs and Weston 2007). In their explanation of business cycles, textbooks also introduce fraud, internal controls and relational databases. Internal controls, such as segregation of duties, authorization, independent verification and access control, safeguard business cycles primarily to prevent asset misappropriation, collusion and fraudulent financial reporting but also errors. Because the business cycles of many firms rely on network technologies, the current curriculum also lists Internet threats and the access control protocols used to combat them. Additionally, to explain digital data storage, textbooks discuss relational databases, Entity-Relationship (E-R) and Resource-Entity-Agent (REA) database modeling and database access controls frequently in the context of Microsoft Access.

A revision of this model and its implementation in current AIS curricula is needed to provide accountants with knowledge concerning key elements of an information system and their relationships, as well as the skills to meet employers' current expectations. With respect to data analytics, the revised model should include the technologies needed to convert structured and unstructured data into useful reports, and the curriculum should provide a conceptual understanding of and practical experience with multiple analytical tools. Although the current strong focus on data analytics within the accounting profession and industry makes this addition to the current model and curriculum worthwhile, additional modifications are necessary for two reasons. First, even an expert in analytical tools is not useful without a thorough understanding of data and the information system, and second, data analyst is not the only IT role that the accounting profession must fill.

Good data analysts understand data and information. The current AIS model includes some of the steps in the information life cycle, but a revised model should adopt the more comprehensive perspective of information scientists. Furthermore, despite the inclusion of some steps in the current model, the curriculum has not taught the principles of the information life cycle, but rather has focused on the general ledger system that comprises business cycles and accounting records. The promotion of the REA database model, which is not a conceptual database model, but rather a static representation of accounting transactions, is evidence of this focus. However, business cycles and accounting records are applications of the information system and not the information system itself. Because it is more difficult to infer correct information system principles from observing a subset of applications than to apply the principles to a business cycle once learned, the revised curriculum should first teach the principles of the information life cycle and then use business cycles as use cases to allow students to understand the application of these principles. Comprehension of these principles is necessary for students to understand a firm's data and informational needs sufficiently to serve as data analysts.

Accountants should also aid in systems design, maintenance and audit. A proper understanding of the information life cycle is essential to these roles, but because technology is the physical makeup of the information system, knowledge of the IT infrastructure is also paramount. The current model includes data storage as a technology, and textbooks discuss accounting journals and ledgers and relational databases as the implementation of data storage. Because the IT infrastructure necessary to support the information system involves more than data storage, the revised model should include additional IT components, and the curriculum should train students to interact with and audit these components. With respect to databases, because E-R modeling is the most widely accepted method for creating conceptual database views, the curriculum should teach this method instead of REA.

IT auditors must also understand internal controls. The current AIS model includes no reference to controls, but the accounting curriculum's treatment of internal controls with respect to business cycles is extensive. However, a recent survey indicates that information security is a primary focus of firm controls, whereas fraud, asset misappropriation and business processes are only of secondary or tertiary significance (E\&Y 2013). Because internal controls should promote information security, the revised model should include them. Additionally, the revised curriculum should direct attention toward controls over information and focus less on controls over business cycles.

Figure 2 displays the model of the Accounting Architecture (AA) framework. We selected the name Accounting Architecture for two reasons. First, by using a name other than Accounting Information Systems, we reinforce the differences between the existing AIS model and this revised model, as well as the differences in approaches to the curriculum. Second, as a prominent software engineer for Fidelity observed, an individual who understands both the business and the information system would constitute an ``accounting architect.''

\InsertGraphic{Figure 2}

The arch visually represents the hierarchy of the components of an information system, and within each group are multiple building blocks that subdivide these components into their relevant constructs. Information is the most important and sits at the top of the arch. Technology and Control are the legs of the arch because they play supporting roles by providing the tools to create, store and analyze data and the controls to maintain data security and integrity. Finally, at the base of the arch is the foundation that comprises the constraints of a useful information system. The individual blocks in each section contain summary topics that are further divisible. The decision to include only summary topics increases the model's readability without restricting the number of potential subtopics within each construct.

This new model extends the existing AIS model in several ways. First, the Information section of the arch portrays the activities in the information life cycle as currently understood by information scientists in three non-linear groups: creation, use and maintenance.\footnote{ARMA International, a professional organization of records and information managers, lists five groups of activities in this life cycle: create, classify, use, retain and dispose (Hoke 2011). Create includes needs assessment, acquisition, conversion and entry; classify comprises classification, storage, organization and indexing; use contains analysis and reporting; retain involves rights management, refreshing and interpretation; and dispose is deletion or, in the case of physical assets which is less relevant in our context, dispose can also involve returning borrowed materials. We combine these five into three constructs for two reasons. First, classification should occur at data creation, and storage, organization, indexing, rights management, refreshing interpretation and disposition are all ongoing maintenance activities (Upward 1996; Corrigan and Sprehe 2010). Second, by combining these five into three, we preserve the life cycle constructs as presented by ARMA while increasing the tractability of our model.} Creation comprises needs assessment, data acquisition, classification, conversion and data entry; use comprises searching, data analytics and reporting; and maintenance involves storage, organization, indexing, rights management, refreshing, interpretation and disposition. While creation activities should occur before use, maintenance activities do not always follow use. It is important to note that use, the keystone of the arch, is the primary purpose driving all the other activities. The ability to retrieve and analyze relevant information is paramount, and the entire goal of the framework. A proper understanding of the process by which data becomes information, and of how information should be managed, are necessary to comprehend the information system. This understanding will help prepare students to develop new systems, as well as support and audit existing systems.

Second, including technology in the framework pairs information life cycle principles with the physical infrastructure that manifests those principles. The IT infrastructure is the application of information system principles, and although the principles apply across multiple different implementations, teaching applications trains accountants to use real-world tools to bring principles into practice.

Third, this model explicitly includes control activities as supporting the information system. Internal controls are paramount to accountants, and a primary focus of control activities is information security.\footnote{The AICPA Trust Services Framework is evidence of accountants' understanding the importance of data security and integrity. We adopt this framework as the model for the Control section.} An understanding of this role of internal controls is necessary in order to train IT auditors and analysts.

Fourth, by including compliance and the business model in the foundation, the model links the information system with accounting rules and business cycles as currently taught in accounting curricula. The ability to perceive this link will allow accountants to view financial, tax and management accounting, audit, the information system and the IT infrastructure as belonging to one unified system instead of disparate sets of regulations, policies and practices.

An additional contribution of the AA framework is a roadmap for curriculum revisions that will align education with practice. Although both the current and the revised models focus on data and information, the current curriculum addresses primarily business cycles. The revised curriculum would redirect attention towards teaching the principles of the information life cycle first and then explaining business cycles as implementations of an information system. Additionally, the revised curriculum would explain internal controls in the context of information security instead of business cycle security, and the curriculum would provide an understanding of and practical experience with enterprise-grade programming languages, operating systems, databases and data analysis tools. Finally, because the AA model encompasses information systems and accounting regulation, a revised curriculum could begin with an introduction to Accounting Architecture even before teaching accounting standards so that all accounting education could be cast in the context of a single unifying framework.

\Section{III.}{Accounting Architecture Framework}
The preceding section compared the existing AIS model and curriculum with a revised model and curriculum of the Accounting Architecture framework. This section provides additional clarity by defining each block and explaining how the individual constructs in the AA model provide accountants with a useful education. We begin with the blocks in the foundation and then present the blocks in the Information, Technology and Control sections of the arch.

\SubSection{Foundation}
The architecture is a function of every block in the framework. The three blocks that establish the foundation are compliance, business model and technological feasibility. The unique role of these is that they set the boundaries of a useful system. They are not sufficient to identify the optimal architecture, but they can rule out a variety of suboptimal solutions. In construction, a building foundation alone cannot determine the best floor plan, but it can preclude a great number of incompatible floor plans. One accounting example is the need to track depreciation of fixed assets. This need provides little guidance towards the ``best'' system, but it does rule out any system that fails to track depreciation. A change in regulation or business model or any technological innovation can necessitate a change in the structure of the accounting architecture, but because these constraints only set boundaries, not every change in them will result in a redesign of the architecture to the extent that any existing setup remains ``in-bounds'' following the change in constraints.

The origin of these constraints can be either external to the firm or internal. Compliance with stakeholders' demands represents an external constraint, whereas the firm's chosen business model creates an internal constraint. Technological feasibility sits in between because it is determined by forces both outside and inside the firm.

\SubSubSection{Compliance}
Not all information systems are subject to regulations. Not even all corporate information systems are subject to regulations. Data confidentiality is an important aspect of information security, and under certain circumstances regulatory bodies dictate confidentiality requirements (e.g., HIPAA, FERPA), but systems design from the perspective of information scientists is not necessarily constrained by regulations. Accounting, on the other hand, is primarily regulation driven. The majority of these regulations stipulate the information that a system should collect and communicate. For publicly traded firms in the US, generally accepted accounting principles and the internal revenue code are regulations that strongly influence architecture, but regulatory compliance certainly extends to private companies, as well as companies outside of the US. Frequently, the question is not whether regulatory compliance applies, but rather which regulations constrain the information that the system must generate.

Compliance also extends beyond regulatory compliance. Other stakeholders may require a system to exhibit certain characteristics. Creditors may request certain reports, or customers may demand specific internal controls before sharing personal or financial information. Although companies may not have a legal obligation to these stakeholders, if a necessary business relationship stipulates how financial information is collected, maintained or communicated, then the information system must reflect those requirements.

Because of the nature of the accounting profession, most curricula teach rules and regulations. However, despite the fact that regulatory compliance is a key driver of IT infrastructure (E\&Y 2013), current courses rarely present compliance in the context of the entire information system. AIS currently addresses the link between compliance and systems architecture to some extent by demonstrating how business cycles create individual accounting numbers, but a more thorough explanation of the influence compliance has on an information system will provide a much needed perspective of systems design through the eyes of an accountant. Including this explanation early in the curriculum can result in synergies between subsequent information systems courses and accounting courses.

\SubSubSection{Business Model}
The business model is the set of business activities in which the firm has chosen to engage. This comprises processes (e.g., sales order processing, cash receipts, payroll, inventory purchases), ownership structure (e.g., public vs. private, LLC vs. S Corporation), industry, organizational hierarchy, as well as anything else that affects business operations, such as firm size, debt structure, globalization and vendor and client relationships. This foundational block is internal because the firm can unilaterally select its business model. It is simultaneously a constraint of the information system because the system is responsible for measuring whichever activities the business model prescribes.

Because the information system can report on the profitability of chosen activities, it can serve to identify future business models. This is not evidence that the information system constrains the business model, rather it implies a feedback loop between the business model and the information system. As the information system reports on the performance and profitability of business activities, managers revise strategic goals and select a new business model. This new business model, in turn, sets new boundaries for the information system. This interpretation differs from the implied perspective that the ERP system should shape the business processes. Equipped with new database and cloud computing tools, firms now have the freedom to select any suitable business model and choose a combination of storage and end-user applications to support those processes.\footnote{Chaotic architecture is further evidence of a departure from the ERP model in that it ask end-users to develop their own modules and interfaces that the system then assimilates.}

In addition to the feedback loop between the business model and the information system, another feedback loop exists between business model and compliance. The business model determines the standards with which the business must comply. For example, because AT\&T is in the telecommunications industry, it is subject to consumer privacy standards as dictated by the FCC.\footnote{http://www.reuters.com/article/2015/04/08/us-at-t-settlement-dataprotection-idUSKBN0MZ1XX20150408} Perhaps the most significant examples of the link between business model and compliance are the accounting regulations associated with choosing to list securities on an exchange and tax compliance based on industry classification and geography.

Accounting education already addresses tax, audit and financial accounting compliance for many different business models. AIS also already introduces various business processes and the data flows used by these processes. However, the business model and its tie to compliance are key drivers of a firm's information needs with respect to both systems design and data analysis. As a result, an explanation of the implications of a chosen business model for informational needs should accompany the introduction of the business model and its effect on regulatory compliance.

Process diagramming is a useful tool for visualizing and communicating business model activities. Current AIS textbooks all explain some form of diagramming (e.g., flowcharts, data-flow diagrams), and the future curriculum should continue this practice for two reasons. First, teaching students to read diagrams facilitates communicating complex business cycles as use cases for information systems. Second, firms in practice continue to use process diagrams for internal purposes.

\SubSubSection{Technological Feasibility}
Unlike regulatory compliance and the business model, technological feasibility is both an internal and an external constraint. The external component is the fact that some systems are simply not possible. Current computing standards establish a frontier beyond which technology cannot operate. For example, quantum computing dwarfs the speed of mechanical computing, but currently, quantum computing exists only in theory. Robotic automation provides another example. Sentient robots may allow for unprecedented efficiency gains from automation, but sentience in robots, the intelligence of Watson notwithstanding, still remains in the realms of Asimov's fantasies.

Technological feasibility, however, extends beyond computational limitations. Frequently the question is not whether a solution could exist, but rather whether it currently exists. Hardware and software solutions can come from vendors or the open source community, or they can be developed internally. Herein lies the internal component of this constraint. The information system cannot rely on any nonexistent hardware or software unless the company agrees to design it or commission it, but the company must choose whether the new business activities or new information, which new hardware or software facilitate, satisfy goals and budgets. This choice gives the company some control over whether a desired architecture is feasible.

\SubSection{Information}
The primary focus of any information system and the cap of this arch is information. Accountants already know what reports decision makers and external stakeholders want. Therefore, the purpose of this framework is not to convince accountants that more or different reports are necessary, but rather to convey an understanding of the process by which data becomes information. Despite information being the end goal of all accounting activity, the principles of information science are foreign to most accountants, and a discussion of the information life cycle is conspicuously absent from the current AIS curriculum. Without knowledge of these principles, accountants cannot properly design, maintain, audit or even understand an information system. The relevant building blocks are not journal entries, ledger accounts, financial statements or any other familiar aspect of double-entry accounting. These are only sample implementations of the principles. Although focusing on accounting records is appropriate for financial accounting courses, the revised AIS curriculum should teach the information life cycle so that students can understand the purpose of an information system.

In our framework, we subdivide the steps of the information life cycle into three blocks: creation, use and maintenance. Each of these blocks comprise multiple activities. This section discusses some of these, while others belong in the control section. Although newer architectures are likely to rely primarily, or even exclusively, on computerized systems, the principles embodied in these three blocks extend to any information system, regardless of form.

The steps in the modern information life cycle are not linear,\footnote{Some information specialists refer to an information continuum, instead of a life cycle (Atherton 1985).} and the management of digital data is an ongoing process, therefore the information system should be complete prior to the first transaction to verify that all data can successfully progress through the cycle to become useful, accurate information. Any attempt to enter data into an incomplete system would be comparable to the attempt to store rainwater with a drain pipe but no barrel. The rain will flow into the pipe, but the lack of storage device renders the water irretrievable. One of the risks of teaching accounting information systems using business cycles is that the sequential nature of the cycles can imply that the implementation of the underlying system may also be piecemeal. By focusing on the principles of the information life cycle before introducing business cycles, accountants will gain an appreciation of the need for a complete information system before processing transactions.

\SubSubSection{Creation}
The first step in data creation is needs assessment to determine informational or reporting needs. The business model and compliance blocks in the foundation are particularly important at this step of the architecture design. Compliance determines what information customers, business partners and regulators will demand, and the business model will identify what information managers and other decision makers will need in order to run a profitable business. Once the informational needs of stakeholders are clear, it is necessary to work backwards to identify which data will serve to create that information. An architect skilled in needs assessment understands what the information system should accomplish.

Because the initial needs assessment predates the system design, data acquisition, the second step in the creation phase, is the first activity when processing an individual transaction. Acquisition is the receipt of data from a counter-party. This step addresses what to receive and how to receive it. Current AIS textbooks identify what to receive for each business cycle, but because the principles of data acquisition apply across multiple business processes, students may benefit more from an explanation of the general principles. Sample business cycles can then serve as an applied setting to observe the principles in action. Additionally, in an increasingly digital economy, a thorough explanation of how to receive data is also important to permit accountants to design systems that maximize the ease of data transfer.

In order for data in a system to be interpretable, it must have certain identifying characteristics. Otherwise, data stores would be nothing more than collections of seemingly arbitrary characters that are neither computer- nor human-readable. Classification is the process of assigning these identifying characteristics. Metadata represents the primary means of classification, especially among digital data. Consistently applied metadata is paramount for searching and data analysis. Additionally, accountants must understand the metadata or data analysis may result in misled inferences.

In addition to proper classification, the data may require conversion prior to entry. Data classification is not optional, but conversion is only necessary when the information system expects the data to have a different format than during acquisition. The resulting uniformity of data formatting facilitates accurate and efficient analysis. A well-designed data acquisition process can alleviate the need for data conversion by reducing or removing incompatibilities between source and target information systems, but because it may be too costly for the counter-party to provide data in a certain format, data conversion allows the firm to accept the cost of formatting the data, instead.

Data entry, when preceded by proper needs assessment, acquisition, classification and conversion, is straight-forward. Data entry can take many forms: manual or automated, continuous or batch, digital-to-digital or paper-to-digital. This step already exists in the current AIS model under Data Collection, and the current AIS curriculum discusses data entry procedures. However, the revised curriculum should place this explanation of data entry in the context of the other steps in the data creation phase.

\SubSubSection{Use}
The pinnacle of the information life cycle is the use of data to disseminate information. Data creation and certain maintenance activities prepare the data for this transition into information. Searching, analysis and reporting are the primary use activities that transform data into useful information. Although these steps apply to the information life cycle in general, in the context of accounting information, these steps jointly constitute business intelligence.

Searching is the method for retrieving data and delivering it for analysis and reporting. Effective searches require common vocabulary and properly assigned metadata. A vocabulary is a set of carefully chosen words and phrases used to identify and distinguish objects in a group, with each object being assigned one or several terms. Subject librarians are typically the best choice for assigning the appropriate terms in a retrieval system's vocabulary.

The two most frequent measures of a successful search are precision and recall. Precision is a measure of the fraction of the total retrieved items that are relevant to the search. Recall is the fraction of all relevant items in the system that are actually retrieved (Manning, Raghavan and Sch\"utze 2008).\footnote{These two concepts mirror Type I and Type II errors, respectively. The assumption of this model is that any retrieved item is either relevant or non-relevant, although modern models also use the concept of ranking.} The ideal system will return the most relevant items with a minimal number of false positives. After proper assignment of metadata, the single most valuable way to increase the effectiveness of a search is proper user training.

Data analytics is the need that has prompted the curriculum review. However, the revised curriculum should not only expose accountants to analytical tools, but also explain the models behind them. The most common programming models for organizing and analyzing Big Data (i.e, data mining) are MapReduce and Online Analytical Processing (OLAP). MapReduce reduces the dimensionality of data to increase interpretability, and OLAP visually portrays aggregated data. In order to recognize the appropriate analytical tool, students must understand what purpose the tools serve.

Reporting is already the primary focus of accounting curricula, and accountants understand the need for compliance reports, including SEC filings and tax forms, as well as internal reports. Because of a focus on the general ledger, the current AIS curriculum provides a narrow perspective of reporting---primarily financial accounting.  The revised curriculum should demonstrate how the information can also generate other compliance reports, and it should also emphasize internal reports. Although the ability to generate all necessary compliance reports is sufficient for an information system to be useful, internal reports of skillfully analyzed data provide decision makers with valuable insights. It is the value of these reports that has spurred Big Data analytics. One important characteristic of useful internal reports is data visualization, and although the current accounting curriculum already addresses many accounting reports, the revised curriculum should introduce data visualization techniques so that systems designers can increase the value and readability of internal reports.

\SubSubSection{Maintenance}
Maintenance governs data storage, organization, indexing and disposition.\footnote{Rights management, interpretation and refreshing are maintenance activities that belong to the control section.} Both the business model and compliance dictate many aspects of a maintenance plan, but some aspects simply constitute best practices.

Storage is the repository that collects all acquired and internally created data. This can be as simple as a filing cabinet for source documents and printed reports or as complex as databases, shared flat-file systems and version control systems for digital data. Regardless of the form of the storage repository, it must be scalable to accommodate future creation activities without necessitating excessive disposition of stored data. Because accounting records are examples of data storage, accountants must understand proper storage principles in order for the measurement of business transactions to progress successfully from source data to financial statements, tax documents and internal reports.

Along with data classification during the creation phase, organization of stored data increases findability by ordering or grouping items by relevant attributes. Indexing also increases findability. Reference manuals frequently include an index as a convenient method for finding information in a book. Any set of physical or digital assets can have an index to allow for more efficient data retrieval. The index consists of a key and a pointer. The key can be any metadata characteristic, and the pointer identifies location. Libraries use catalogs as indexes to allow for faster book retrieval, and databases use indexes to reduce system resources needed to run queries.

The revised curriculum should include data organization and indexing principles because findability is one of the most important attributes of useful data. In the current Big Data environment, no one can accidentally find one specific data item among petabytes of data across multiple servers, hard drives and even data warehouses. This is not only a problem of digital data. A brokerage firm had developed the practice of storing customer documents by date in a box in a closet. When the closet was full, the box was transferred to a storage unit. During an audit, the firm spent countless hours retrieving and reviewing documents from that storage unit in order to find a small number of files needed for the audit. Proper organization and indexing decrease search time and increase both data usefulness and compliance.

Disposition occurs at the end of the information life cycle. Metadata should include a predetermined end of life similar to the useful life associated with depreciable assets. The concept of asset disposal is familiar to accountants. Data is also an asset, and students should be familiar with this final step in the information life cycle.

\SubSection{Technology}
The first leg of the arch is technology, or the IT infrastructure. The four components of the infrastructure are hardware, software, storage and services. Unlike the Information section of the architecture which describes the information life cycle for digital and non-digital systems alike, the use of technology is unique to computerized systems. The proper treatment of data in an information system is requisite, but the use of technology is optional. The applicability of this leg of the arch will vary based on the desired reliance on information technology. As most systems are not completely void of computers, some manifestation of these principles will usually apply.

The current AIS curriculum includes little discussion of information technology despite employers desiring to hire accountants with IT knowledge. As a result of the lack of IT training, students gain a poor understanding of operating systems, server and network equipment and enterprise-grade databases and software. This siloed approach to education has precluded accountants from participating in IT audit functions, systems design and maintenance, database administration and end-user software development. The goal of adding IT instruction to the curriculum is not to encourage accountants to become engineers, but rather to allow accountants to bridge the gap between business knowledge and information technology.

\SubSubSection{Hardware}
Accountants should be familiar with the basic building blocks of a computer: central processing units (CPUs), random-access memory (RAM), motherboards, graphical processing units (GPUs) and storage drives. Additionally, the curriculum should explain the trade-offs between fat clients (i.e., workstations) and thin clients as end-user computing devices and how the decision to employ one set over the other affects the role of and need for server hardware. The cyclical trend of thin and fat clients renders historical context particularly useful here.

Because the information system can rely on very large computing devices, accountants should certainly have a greater understanding of servers. The building blocks of these are not different from the building blocks of a personal computer, but the implementation can vary in important ways. For example, although personal computers usually function independently, servers dedicated to the analysis of vast amounts of data may work in a cluster to share processing power. Firms that wish to analyze large structured and unstructured datasets would be better served by accountants who understand the hardware necessary to facilitate the collection, storage and analysis of this data.

The network infrastructure allows everything to communicate with everything else. Many, if not most, information systems rely on network technologies to collect data from external stakeholders, such as customers and vendors, and to communicate data and reports to employees and decision makers. Because of the nature of Big Data, network-connected data stores have become increasingly important, and a systems designer should recognize the requisite hardware to support external and internal communication.

\SubSubSection{Software}
Three aspects of software make it a more important focus than hardware. First, for obvious reasons, hardware is impotent without software. Second, virtualization has allowed software to act as hardware, reducing the cost of and reliance on hardware in the IT infrastructure. Third, relatively little variation exists between the hardware components of various networks, whereas software, especially in-house developed software, can offer unique capabilities and benefits.

The self-evident tie between hardware and software is manifest in the ambiguity of the word ``server.'' Although common, the definition of server as server hardware is less appropriate than the definition of server as software that provides a service. For example, a web server is not a computer that hosts a web page, but rather the software, such as Apache HTTP Server, on that computer. Based on this definition, any networked computer can be a server, and indeed this is true. Correct understanding of that fact alone would provide students with increased ability to understand and interact with the information system, especially given the current reliance on virtualization. Virtualization allows software to mimic the functionality of hardware in order to duplicate hardware resources, and it is an important component of cloud computing.

The operating system (OS) is the first step in using software to harness hardware functionality. Microsoft Windows and Apple OS X are the two most common operating systems for workstations,\footnote{http://www.netmarketshare.com/operating-system-market-share.aspx?qprid=8\&qpcustomd=0}  but many other OSes exist, and the use of other, less commonly advertised operating systems (i.e., UNIX, Linux and BSD) on servers and on thin clients dwarfs the adoption of those two.\footnote{67 percent of web servers use BSD or Linux; 59 percent of mobile devices use Android (Linux) and 27 percent use iOS (BSD); and 95 percent of supercomputers run Linux. Linux is also a popular choice for firmware (i.e., small embedded operating systems for network hardware, printers, televisions, game consoles, automobiles, jets, etc.)} Most accountants are familiar with Microsoft Windows or Apple OS X, but because enterprise-grade information systems more frequently rely on Linux, BSD or UNIX, knowledge of these is essential for systems design, maintenance and audit.

Although the operating system is the most basic implementation of software, it is not as important as the other software in facilitating the knowledge-creation process. This software may take the form of web applications, which are particularly useful for acquiring data from external stakeholders, or business intelligence tools. Software may be purchased, developed in-house or open source, and any software selection process should consider each of these outlets. Many proprietary business intelligence solutions exist (e.g., SAS, Tableau, Splunk, Caseware IDEA, ACL), and the curriculum should provide students with practical experience with several of these.

The current frontier of technological innovation will strongly influence whether to develop software in-house. In fact, this component of the Accounting Architecture is more subject to the technological feasibility constraint than any other because a unique business model will require unique software to generate unique information. If no such software exists, the only choices are to forego the information or to innovate by developing the software in-house or commissioning custom software. Cost-benefit analysis will determine whether the information is worth the investment in the software development. In order for accountants to assess the net benefit of developing software in-house, they must have some understanding of the software development process. Just as with IT hardware, the specifics of software development can quickly become too technical for accountants, but the curriculum should include some fundamental concepts, such as the software development life cycle---planning, analysis, design, implementation and maintenance---development methods (e.g., waterfall, agile) and common programming languages and quality assurance tools. Accountants familiar with software development provide a much-needed link between business and IT functions.

Open source software offers a recent alternative to proprietary or in-house developed software. The idea behind open source software is that the developer makes the source code available so that users can modify and freely redistribute the code. According to a recent survey, 78 percent of firms use open source software because it is more secure, scales better and deploys more easily.\footnote{http://www.zdnet.com/article/its-an-open-source-world-78-percent-of-companies-run-open-source-software} The most popular Big Data analysis tools are open source. Because of the importance of open source software for data analytics specifically and for the information system in general, accountants should recognize prominent open source projects, understand how to obtain open source software and know the history of the current open source movement.

\SubSubSection{Storage}
Almost as important as software is the storage solution. Storage is neither solely hardware- nor solely software-driven. The most relevant consideration is the choice of database structure, but decisions regarding file systems are also impactful.

Many AIS textbooks and courses presently teach relational databases (RDBMS) and SQL, although some have recently dropped this topic. The proposed curriculum not only reverses this departure from teaching databases, but rather underscores these concepts. Because of the difference in administration techniques, whenever possible the curriculum should demonstrate and provide experience with multiple RDBMS. Additionally, students should gain more than a cursory knowledge of SQL. Although PWC views Microsoft Access as a valuable legacy technology (PWC 2015), it is not an enterprise-grade database solution. Familiarity with other, more powerful databases will better prepare students for systems design, maintenance, audit and especially data analytics.

While relational databases will continue to be relevant, recent critiques have promoted NoSQL (Not Only SQL) databases as an alternative because of it is better equipped to handle Internet traffic and Big Data. Web technologies and data analytics are clearly significant information system components, and the curriculum should expose accountants to example NoSQL databases, as well as JavaScript.

Database storage repositories comprise more than RDBMS or NoSQL databases. Version control systems combine files from multiple owners into one centralized storage repository. The repository allows any authorized user to check out a copy of the files, store the files locally, edit local copies and check the changes back in. Version control systems are most popular among software developers because they allow multiple programmers to edit the same files simultaneously and then merge their respective changes into the shared master file. Knowledge of version control systems assists accountants in understanding in-house software development, as well as open source projects.

File systems are also important for multiple reasons. First, all data, including physical databases, resides in a file system. As a result, knowledge of file systems is as fundamental as knowledge of operating systems. Second, databases are not the only manifestation of shared storage. Shared file systems are also prevalent. Third, as newer digital systems increase their similarity to database systems by adding transaction logs and distinct physical and conceptual views, understanding of either one synergistically improves understanding of the other.

\SubSubSection{Services}
Not every part of the IT infrastructure is necessarily managed in-house. Internet service providers provide network hardware and connectivity to external services. However, a far more prominent aspect of outsourced IT services is cloud computing.

Cloud computing is the extension of the traditional model of a dumb client connecting to a server, but in its current form, the server can be anywhere in the world, it can accommodate multiple clients simultaneously and the clients can do more than simply display a command line. Cloud computing is also the extension of the ERP model, perhaps in the same sense that ERP was the extension of MRP II, and any knowledge regarding cloud computing is valuable towards understanding existing ERP implementations. However, although the current AIS curriculum emphasizes ERP, the revised curriculum should focus on the more general cloud computing model.

Cloud computing has three service levels: Infrastructure as a Service (IaaS), Platform as a Service (PaaS) and Software as a Service (SaaS). Each service level includes different components of the IT infrastructure, and each exists in the form of both public and private clouds. Public clouds are clouds that vendors provide to customers. Private clouds, on the other hand, have a similar structure except that the firm hosts the cloud for itself, which can reduce cost and increase security.\footnote{The leader in enterprise-grade private cloud offerings is CloudStack, and demand for expertise with CloudStack is perhaps more pronounced than with any other component of the IT infrastructure.} A third, hybrid model that combines the security aspects of private clouds in a public cloud environment is the virtual private cloud.

\SubSection{Control}
Information and technology work together to create, maintain and analyze data. Together, these form a complete system, but without control this system has risks. Control preserves the integrity of the data by avoiding unauthorized access, loss and error. Because perfect prevention is not feasible or even possible, a properly controlled architecture should also detect and correct problems that it failed to prevent. No system is inherently safe, including small systems that benefit from security through obscurity. Even personal email accounts are targets of attacks, and no amount of obscurity can prevent data loss or errors. Whether the likelihood of attack, loss or error is large or small, the consequences of these are too dire to forego proper controls.

Most current curricula already include an extensive treatment of internal controls. The purpose for discussing controls in this paper is two-fold. First, because of the importance of control, any model of accounting information systems should reference it. Second, the primary focus of controls in an information system is information security, but current textbooks emphasize business process security. The proposed curriculum redirects attention towards controls over information. Accountants have considerable experience with internal controls, so it is not necessary to devise an alternative control framework in order to revise the curriculum. As a result, the control section of the arch mirrors the Trust Services Framework\footnote{Some AIS textbooks already discuss the Trust Services Framework, which reinforces the value of this framework for the accounting profession.} for systems reliability: security, availability, processing integrity and confidentiality (AICPA 2014).\footnote{The traditional representation of the Trust Services Framework includes privacy as an additional principle. However, because the treatment of confidentiality and privacy is similar, we combine them into one block and retain confidentiality as the name for both constructs. This is consistent with the tenor of Coe (2005).} These four principles combine to support information security and data integrity and to prevent data leakage.

\SubSubSection{Security}
The role of security is to control physical and digital access through permissions, encryption and authentication. Role-based permissions determine which employees have access to resources. Encryption renders content unreadable by unauthorized users. Authentication allows individuals to identify themselves as authorized users via something they know (i.e., password), something they have (i.e., security token) or something about them (i.e., biometrics). Multi-factor authentication combines multiple authentication methods together to increase security. Current threats to information security frequently necessitate multi-factor authentication.

Because information security is a high priority, this is the foundation of the Trust Services Framework. Security facilitates the other Trust Services principles. Accountants should have up-to-date knowledge of security protocols, as well as current threats to information security. Although high-profile attacks on Home Depot, Target and other retailers are salient examples of threats, accountants should also know less well-publicized security risks and software vulnerabilities, such as Heartbleed, Shellshock and Venom. The curriculum must constantly evolve to introduce accountants to current threats and technologies. Additionally, accountants should gain hands-on experience with firewalls, permissions, authentication methods, access log analysis, software patching and other security practices.

\SubSubSection{Availability}
In order for an information system to serve any purpose, it must be available. Two threats to availability are downtime and data loss. System downtime results from loss of power or network connectivity or from the need for system maintenance. Data loss can occur because of software or storage corruption, hardware destruction or theft. Although as with other control activities, prevention is good, with respect to availability, the stronger focus is on correction through backups and hardware and network redundancy.

Data interpretation, part of maintenance in the information life cycle, also supports data availability. As data becomes older, the software and hardware necessary to retrieve the data become outdated. When this occurs, data may become unreadable. Data interpretation involves storing data with the hardware and/or software necessary to read it.

In an enterprise-grade information system, especially a system that interfaces with customers, vendors or other external stakeholders, downtime is generally unacceptable. Furthermore, regulatory compliance may not tolerate the loss of accounting data. Accountants should understand technologies and services that support redundancy, as well as accepted backup, restoration and interpretation practices. These skills are relevant for systems design, maintenance and audit.

\SubSubSection{Processing Integrity}
Information security is a top priority, but data integrity also is paramount. The purpose of any information system is to create data and convert it into information. If the data is inaccurate, then the information will be valueless, or worse, detrimental. Controls protect data integrity during creation to ensure that data in the system is correct and during maintenance to ensure analysis of current, valid data.

These first three control components are vital to any information system, and without all of them the information system is at risk of not serving a useful purpose. In order to train accountants to maintain data integrity, the curriculum should provide students with an understanding of software development and quality assurance. Software code governs data entry and processing, and accountants should be able to audit code to determine whether proper controls exist. Additionally, refreshing provides stores with current data for subsequent analysis, and accountants should know best practices for data refreshing.

\SubSubSection{Confidentiality}
Information systems frequently contain sensitive information. This information may be legal documents, trade secrets or other forms of intellectual property. Confidentiality protects data from unauthorized uses. The first step in enforcing confidentiality is rights management. All content that is not in the public domain is subject to some license. In order to comply with copyright and copyleft,\footnote{With the advent of open source, organizations have developed content licenses that govern fair use. Copyleft dictates that a work under its license is free and that any derivative works must also be free. For more information, see http://www.gnu.org/copyleft/.} any information acquired or created must have the appropriate rights metadata accessible.\footnote{http://www.copyright.gov/title17/92chap12.html\#1202} Rights metadata lists content owner, license type, permitted uses and license expiration. The next step is to use security tools---permissions, encryption, authentication---as well as disposition when licenses expire to preserve confidentiality. Regulatory and contractual compliance strongly influence confidentiality requirements, and accountants should understand rights management in order to comply with these.

\Section{IV.}{Curriculum Revisions}
The preceding section introduced the proposed Accounting Architecture model, described the blocks in the model's arch and explained consequent modifications to the curriculum. This section presents two proposals for implementing these revisions into accounting education by identifying specific courses to teach the requisite skills. The first proposal addresses upper-level and graduate courses only, whereas the second proposal builds on the first by suggesting changes to lower-level accounting courses.

\SubSection{Upper-level Courses}
The two key components of the revision to upper-level courses are a restructured undergraduate course in AIS and a new master's track. Because other disciplines have expertise in certain components of the AA framework, accounting departments will not need to create new courses to address every topic, but rather simply identify existing courses in other departments and incorporate them into the course plan.

\SubSubSection{Introduction to Accounting Architecture}
The first curriculum change involves replacing the current AIS course with an introduction to the AA framework. This course should present the AA model, and expound upon each of its building blocks. Because one course is not sufficient to provide students with expertise in any of the constructs in the model, the goal of this course must be to instill interest in and familiarity with each of the topics. Additionally, the course should provide practical experience with the components of the technology section by encouraging students to build a computer, install an operating system, create a virtual machine, administrate and query a database, format a partition, host a website, use MapReduce on unstructured data, manage a private cloud, implement a firewall and set up a local-area network. Not all of these experiences have equal value, but each exposes the students to technological components of most enterprise-grade information systems.

Additionally, with respect to relational databases, the curriculum should discontinue its explanation of database modeling using the Resource-Entity-Agent (REA) framework for two reasons. First, the Entity-Relationship (E-R) model is the accepted standard for relational databases to such an extreme that many professionals in the information systems, IT and computer science disciplines are not aware of the REA model. Second, the REA model is not a conceptual view of a database and requires an E-R diagram as an intermediate step to transition from REA model to database schema. The REA model is a static representation of a business transaction similar to a flowchart or a data-flow diagram. As such, it may be a useful tool for visualizing business transactions, but flowcharts and data-flow diagrams are more widely accepted alternatives for this. Because the REA model cannot represent databases or transactions better than the accepted tools, the decision to redirect the curriculum away from it towards alternative topics would be more pedagogically efficient.

Business cycles also represent an application of the information system, and although they should not be the focus of the course, walking through sample processes can provide students with a contextual perspective of information system principles and practices. However, because different firms implement their processes idiosyncratically, students should understand that, for example, the revenue cycle as explained is merely a sample manifestation of that cycle and not the only, or even the only correct, manifestation of that business cycle. Historically, flowcharts and data-flow diagrams have aided in the portrayal of business cycles, and the revised course curriculum will continue to benefit from the use of these tools.\footnote{Just as the curriculum must change to accommodate new technologies, teaching process diagramming must also change to reflect current practices. Historically, flowcharts and data-flow diagrams have been popular diagramming tools, but current Business Process Model and Notation (BPMN) has grown in popularity, and it may be more appropriate to teach BPMN than flowcharts or DFDs.}

Because of the rapidly changing landscape of technology and information systems, a traditional textbook may not be the best foundation for this course. Instead, news articles can educate individuals regarding new technologies and adoption trends, and other publications (e.g., COBIT 5) can explain the principles of the information life cycle and internal controls.

\SubSubSection{Graduate Courses}
Some universities have begun investigating the creation of accounting master's tracks or emphases in either data analytics or IT audit. An introduction to the AA framework exposes students to aspects of both fields, as well as systems design. Despite the brevity of many master's programs, the overlap in content among these three areas can allow for a single master's track in accounting architecture. In addition to the overlap in content, a commonality in professional roles encourages the implementation of a single track for AA. For example, internal audit departments use data analytics to detect problems. The link between systems design and systems audit is even more intuitive. The goal in both cases is to determine how a system should function. Accountants are already familiar with this duality because financial preparers need the same knowledge as financial auditors, and tax preparers need the same knowledge as tax auditors.

Because the information system provides accounting information, students should gain sufficient understanding of accounting. Courses in advanced financial accounting, advanced cost accounting and corporate taxation provide a foundation of accounting knowledge. Paired with these, advanced audit increases students' understanding of regulatory compliance.

In addition, students should have a course in relational databases and a course in advanced data management topics, including NoSQL, JavaScript, data warehousing and online transaction processing (OLTP). Many information systems departments teach relational databases, but thorough treatment of advanced data management topics is less frequent and may require the development of a new course.

The final portion of the curriculum is a course in business intelligence and a course in data analytics. These address the use of information, and they are the skills most in demand. Information systems departments frequently teach business intelligence, and in the current focus on Big Data, some departments have also introduced a course in data analytics. Because accountants have specific informational needs, collaboration between accounting and information systems departments in determining the topics in a data analytics course may be beneficial.

These courses would endow students with significant knowledge relevant to systems design, systems audit and data analytics, but because information systems vary by business model and regulatory environment, some accountants may benefit from optional courses in certain special topics. Small businesses rely more heavily on ``legacy technologies,'' such as Excel, Access, QuickBooks or Great Plains. A course providing practical experience with a number of these products would prepare students to work in small business. Additionally, because internal controls provide information security and satisfy regulatory compliance, a course specifically on internal controls may aid students who wish to focus on the control aspect of information systems.

This proposal provides an overview of the courses relevant to preparing a student to become an accounting architect. Because of the dynamic nature of information technology, the list of relevant topics and skills will certainly change in the future, and the curriculum should be flexible enough to accommodate future demands.

\SubSection{Lower-level Courses}
The preceding sections introduce one undergraduate course and a subsequent master's track in accounting architecture. These courses will prepare students to enter the profession and serve as IT auditors, data analysts or systems designers. However, the AA framework also has implications for early accounting courses. Currently, introductory accounting courses teach financial accounting regulations and managerial accounting formulae without first providing a perspective of the information system as a whole. Without this perspective, accounting standards can seem obtuse, arbitrary or irrelevant. The AA framework places accounting information, regulatory compliance, the business model and the information system in a unifying context. If, prior to teaching any regulations, the curriculum first introduced the AA framework, students would be better able to grasp the importance of accounting standards. One option would be to include the AA framework as an introductory module in the first accounting course. Another option would be to implement an introduction to Accounting Architecture as the first accounting course and a precursor to fundamentals of financial or managerial accounting. This introduction would not include the depth or the practical experience of the upper-level course presented earlier, but rather would briefly survey the constructs in the arch, discuss sample business processes to demonstrate how the information system might work in practice and present a roadmap of accounting courses and their relation to the model. This course would give students a better understanding of the accounting curriculum and may increase interest in the accounting profession.

In keeping with this idea, it may be important to introduce the AA framework at the undergraduate level in the context of a rapid apllication tool such as Oracle.  Pairing the software development life-cycle to the AA framework early by developing custom apllications that reflect real-world accounting problems will create concrete examples of concepts that, for the unitiated, are often abstract and difficult to grasp.  Additionally, such a class would introduce the relational database and teach students the fundamentals of SQL.  The breadth of this class would likely require two semesters.  

Introducing students to procedural and object oriented programming would also serve a useful role in expanding their understanding of information systems.

\Section{V.}{Conclusion}
Employers increasingly demand new skills in IT audit, systems design and data analytics. The current AIS model and most accounting curricula have focused on business processes since the inception of ERP, and a modification of the curriculum is necessary to train accountants in these new areas. The Accounting Architecture framework provides a model and revised undergraduate and graduate curricula that direct attention away from business cycles and toward the information life cycle, information technology and information security. This new framework gives a unifying perspective of accounting that can help students better understand information, the information system and how accounting standards affect the information needs of the firm. Moreover, the proposed curriculum does more than explain the general ledger system. It prepares students for a timely career as accounting architects.

The fields of MIS and AIS already have considerable overlap. This overlap is the result of all internal and external business reports originating from the same information system, and these curriculum revisions will further train accountants in responsibilities that MIS graduates have traditionally assumed. Employers have requested that accountants assume these responsibilities, and as the custodians of business information, accountants are the best candidates for managing the entire information system, not just the general ledger.

\newpage{}

\centerline{\textbf{\MakeUppercase{Works Cited}}}
\smallskip{}
\singlespace{}
\Reference{AICPA. 2014. \emph{Trust Services, Principles, Criteria, and Illustrations.} AICPA}

\Reference{Atherton, J. 1985. From Life Cycle to Continuum: Some Thoughts on the Records Management-Archives Relationship. \emph{Archivaria} 21: 43-51.}

\Reference{Coe, M.J. 2005. Trust Services: A Better Way to Evaluate IT Controls. \emph{Journal of Accountancy} 199 (3): 69.}

\Reference{Corrigan, M., and J.T. Sprehe. 2010. Cleaning Up Your Information Wasteland. \emph{Information Management Journal} 44 (3): 26-30.}

\Reference{Dederer, M.G., and A. Dmytrenko. 2015. 8 Steps to Effective Information Lifecycle Management. \emph{Information Management Journal} 49 (1): 32-35.}

\Reference{Deloitte. 2013. Technology Trends: Elements of Postdigital. Available at:\\http://www2.deloitte.com/content/dam/Deloitte/us/Documents/technology/\\us-cons-tech-trends-2013.pdf}

\Reference{Ernst \& Young. 2013. EMEIA FSO IT Risk Management Survey: Managing IT Risk in a Fast-changing Environment. Available at: http://www.ey.com/Publication/\\vwLUAssets/Managing\_IT\_risk\_in\_a\_fast\_changing\_environment/\$FILE/\\IT\_Risk\_Management\_Survey.pdf}

\Reference{Franks, P.C. 2013. \emph{Records and Information Management.} American Library Association.}

\Reference{Hoke, G.E.H. 2011. Records Life Cycle: A Cradle-to-grave Metaphor. \emph{Information Management Journal} 45 (5): 28-31.}

\Reference{Jacobs, F.R., and F.C. Weston, Jr. 2007. Enterprise Resource Planning (ERP)---A Brief History. \emph{Journal of Operations Management} 25 (2): 357-363.}

\Reference{Manning, C. D., Raghavan, P., and H. Sch\"utze. 2008. \emph{Introduction to Information Retrieval.} Cambridge University Press.}

\Reference{PricewaterhouseCoopers. 2015. Data Driven: What Students Need to Succeed in a Rapidly Changing Business World. Available at: http://www.pwc.com/us/en/\\faculty-resource/assets/PwC-Data-driven-paper-Feb2015.pdf}

\Reference{Scheer, A., and F. Habermann. 2000. Enterprise Resource Planning: Making ERP a Success. \emph{Communications of the ACM} 43 (4): 57-61.}

\Reference{SINTEF. 2013. Big Data, For Better or Worse. Available at: http://www.sintef.no/\\home/corporate-news/big-data--for-better-or-worse/}

\Reference{Upward, F. 1996. Structuring the Records Continuum Part One: Postcustodial Principles and Properties. \emph{Archives and Manuscripts} 24 (2): 268-285.}

\newpage{}

\begin{figure}[!h]
\begin{center}
\textbf{\MakeUppercase{Figure 1}}\\[.2in]
\textbf{Model of the Components of an Information System According to Existing AIS Curricula}\\[.4in]
\includegraphics[scale=0.8]{Legacy.png}
\end{center}
\end{figure}
\newpage{}

\begin{figure}[!h]
\begin{center}
\textbf{\MakeUppercase{Figure 2}}\\[.2in]
\textbf{Model of the Components of an Information System According to Accounting Architecture}\\[.4in]
\includegraphics[scale=0.8]{Arch5.png}
\end{center}
\end{figure}

\end{document}
